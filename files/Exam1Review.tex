\documentclass[wide,pal,11pt]{pajarticle}

\title{Exam 1 Review}
\date{BIOE 498/598, Spring 2020}

\begin{document}

\maketitle

\begin{enumerate}
	\item Hypothesis testing: interpretation of the $t$-test
	\item Linear Modeling
	\begin{itemize}
		\item Coefficients and residuals
		\item Intercepts
		\item Structure of the design matrix
		\item Significance testing of effects
	\end{itemize}
	\item Experiment Design Terminology
	\item Completely Random Designs
	\begin{itemize}
		\item One Hot encoding
		\item Eliminating degeneracy by dropping levels
		\item Designing contrasts
		\item Testing all contrasts and Tukey's HSD
	\end{itemize}
	\item Logistic regression --- how it differs from linear regression
	\item Factorial Designs
	\begin{itemize}
		\item Advantages over one-at-a-time
		\item Differences in interpretation of effects from one-at-a-time
		\item Rank and power
		\item Interaction plots
	\end{itemize}
	\item ANOVA
	\begin{itemize}
		\item Meaning of $SS_\mathrm{total}$, $SS_\mathrm{explained}$, and $SS_\mathrm{residual}$.
		\item Calculating degrees of freedom for each $SS$.
		\item $F$-statistic: meaning and calculation
		\item ANOVA as $F$-statistic on models with single variables.
	\end{itemize}
	\item Power Analysis
	\begin{itemize}
		\item Standard deviation vs. standard error
		\item Calculating $n$ to resolve an effect size
	\end{itemize}
	\item Fractional Factorial Designs
	\begin{itemize}
		\item Effect Sparsity Principle
		\item Hierarchical Ordering Principle
		\item Base designs and intentional confounding
		\item Generators and defining relations
		\item Confounding or alias structure
		\item Resolution and minimum confounding
		\item Full factorial designs embedded in fractional designs
		\item Half-normal plots to interpret effects
		\item Clearing effects by foldover or mirror image designs
	\end{itemize}
	\item Alternative to Fractional Designs
	\begin{itemize}
		\item Plackett-Burman Designs: creating, complex aliasing, and hidden projection
		\item Regression by subset selection
		\item Orthogonal Arrays for multilevel factorials
	\end{itemize}
\end{enumerate}

\end{document}