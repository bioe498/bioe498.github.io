\documentclass{beamer}

\usepackage{lads}
\setbeamertemplate{navigation symbols}{}

\title{Mixture Designs}
\date{}
\author{BIOE 498/598 PJ}


\newcommand\half{${\textstyle \frac12}$}
\newcommand\third{${\textstyle \frac13}$}
\newcommand\twothird{${\textstyle \frac23}$}
\newcommand\tq{$\quad$}

\begin{document}

\maketitle

\begin{frame}
\frametitle{Mixtures}

Some materials or reagents are \textbf{mixtures}, or combinations of different proportions of a fixed number of ingredients.

Examples of mixtures include:
\begin{itemize}
	\item Polymers
	\item Metal alloys
	\item Fabric blends
	\item Cell culture media
\end{itemize}
 
Studying mixtures requires special designs and models.
\end{frame}

\begin{frame}
\frametitle{Defining Mixtures}
 
A mixture has $k$ components, each with proportion $x_i$, where
\[ 0.0 \le x_i \le 1.0 \]
and
\[ \sum_{i=1}^k x_i = 1.0 \]

\pause
For example, a three-ingredient mixture ($k=3$) is defined by
\begin{align*}
  0.0 \le x_1 &\le 1.0 \\
  0.0 \le x_2 &\le 1.0 \\
  0.0 \le x_3 &\le 1.0 \\
  x_1 + x_2 + x_3 &= 1.0
\end{align*}
\end{frame}

\begin{frame}
\frametitle{Why not factorial designs?}

The summation constraint $\sum_i x_i=1$ prevents us from using factorial designs. Let's imagine a factorial design for a three-ingredient mixture with levels $- = 0.0$ and $+ = 1.0$. The runs are

\begin{center}
\footnotesize
\begin{tabular}{ccc|ccc|c}
	1 & 2 & 3 & $x_1$ & $x_2$ & $x_3$ & $\sum_i x_i$ \\
	\hline
	 $-$ & $-$ & $-$ & 0.0 & 0.0 & 0.0 & 0.0 \\
 	+ & $-$ & $-$ & 1.0 & 0.0 & 0.0 & 1.0 \\
 	$-$ & + & $-$ & 0.0 & 1.0 & 0.0 & 1.0 \\
 	$-$ & $-$ & + & 0.0 & 0.0 & 1.0 & 1.0 \\
 	+ & + & $-$ & 1.0 & 1.0 & 0.0 & 2.0 \\
 	+ & $-$ & + & 1.0 & 0.0 & 1.0 & 2.0 \\
 	$-$ & + & + & 0.0 & 1.0 & 1.0 & 2.0 \\
 	+ & + & + & 1.0 & 1.0 & 1.0 & 3.0
\end{tabular}
\normalsize
\end{center}
 
Only three of the eight runs satisfy the summation constraint!
\end{frame}

\begin{frame}
\frametitle{Models for Mixture Experiments}

The standard first-order model for a factorial experiment with three factors is
\[ y = \beta_0 + \beta_1x_1 + \beta_2x_2 + \beta_3x_3 + \epsilon \]
This model is not appropriate for mixture experiments because of the summation constraint.

\medskip
Instead, we can use two alternative models:
\begin{enumerate}
	\item The slack variable model
	\item The Scheff\'{e} model
\end{enumerate}

\end{frame}

\begin{frame}
\frametitle{The slack variable model}	

Let's start with the standard FO model
\[ y = \beta_0 + \beta_1x_1 + \beta_2x_2 + \beta_3x_3 + \epsilon \]
By the summation constraint, $x_3 = 1 - x_1 - x_2$. Substituting, we see that
\begin{align*}
 y &= \beta_0 + \beta_1x_1 + \beta_2x_2 + \beta_3(1 - x_1 - x_2) + \epsilon \\
  &= (\beta_0 + \beta_3) + (\beta_1 - \beta_3)x_1 + (\beta_2 - \beta_3)x_2 + \epsilon \\
  &= \beta_0' + \beta_1'x_1 + \beta_2'x_2 + \epsilon
\end{align*}

\pause
The effect of factor $x_3$ is folded into the intercept and the other effects. When using a slack variable model we choose $x_3$ to be the least important factor, but the effects remain confounded.
\end{frame}

\begin{frame}
\frametitle{The Scheff\'{e} model}

The Scheff\'{e} model takes a different approach. Rather than substitute for $x_3$, we substitute for the intercept (the 1 multiplied by $\beta_0$) using the constraint $x_1 + x_2 + x_3 = 1$:
\begin{align*}
 y &= \beta_0(1) + \beta_1x_1 + \beta_2x_2 + \beta_3x_3 + \epsilon \\
  &= \beta_0(x_1+x_2+x_3) + \beta_1x_1 + \beta_2x_2 + \beta_3x_3 + \epsilon \\
  &= (\beta_0 + \beta_1)x_1 + (\beta_0 + \beta_2)x_2 + (\beta_0 + \beta_3)x_3 + \epsilon \\
  &= \beta_1^*x_1 + \beta_2^*x_2 + \beta_3^*x_3 + \epsilon
\end{align*}

\pause
The effects in a Scheff\'{e} model are clean, but they have a special interpretation. The coefficient $\beta_i^*$ is the expected response for a pure mixture with $x_i=1.0$ and all other ingredients set to zero.
\end{frame}

\begin{frame}
\frametitle{The Simplex-Lattice Design (SLD)}

A Simplex-Lattice Design SLD$\{k,m\}$ studies mixtures with $k$ ingredients set at $m+1$ equally-spaced levels. The design uses all combinations of the ingredient levels.

\small
\medskip
\pause
\textbf{SLD$\{3,1\}$: Levels 0, 1}
\begin{center}(1,0,0) \tq (0,1,0) \tq (0,0,1)\end{center}

\medskip
\pause
\textbf{SLD$\{3,2\}$: Levels 0, \half, 1}
\begin{center}
(\half,\half,0) \tq (\half,0,\half) \tq (0,\half,\half) \tq + SLD$\{3,1\}$
\end{center}

\medskip
\pause
\textbf{SLD$\{3,3\}$: Levels 0, \third, \twothird, 1}
\begin{center}
(\third,\twothird,0) (\twothird,\third,0) \tq
(\third,0,\twothird) (\twothird,0,\third) \tq
(0,\third,\twothird) (0,\twothird,\third) \tq
(\third,\third,\third)  + SLD$\{3,1\}$
\end{center}
\normalsize

\pause
\medskip
\includegraphics{SLD.png}
\end{frame}

\begin{frame}
\frametitle{The Simplex-Centroid Design (SCD)}

A downside of the SLD$\{k,m\}$ design is there are no interior points until $m\ge3$. An alternative is the SCD$\{k\}$ design which includes
\begin{itemize}
	\item All $k$ pure mixtures: (1,0,$\ldots$,0)
	\item All binary combinations at \half: (\half,\half,0,$\ldots$,0)
	\item All trinary combinations at \third: (\third,\third,\third,0,$\ldots$,0)
	\item $\vdots$
	\item The single $k$-nary mixture: ($\frac1k$,$\frac1k$,\ldots,$\frac1k$)
\end{itemize}

\pause
\medskip
\includegraphics{SCD.png}
\end{frame}

\begin{frame}
\frametitle{Choosing the SLD or SCD}

\begin{itemize}
	\item The SLD$\{k,m\}$ model allows fitting an $m^\text{th}$ order model ($1=$ first order, $2=$ quadratic). The SCD$\{k\}$ can fit a $k$-order model with up to a $k$-way interaction term.
	\item In a SCD model, no single ingredient is run at a proportion $\ge$\half.
	\item Given the same number of runs, the SLD has better coverage of the boundary of the simplex while the SCD has better coverage of the interior.
\end{itemize}	
\end{frame}


\end{document}
