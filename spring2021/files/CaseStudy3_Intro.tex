\documentclass[9pt]{beamer}

\beamertemplatenavigationsymbolsempty
\renewcommand\mathfamilydefault{cmr}

\usepackage{pajmath}
\usepackage{booktabs}
\usepackage{colortbl}

\title{Case Study 3: The Trebuchet}

\author{BIOE 498/598 PJ}
\date{Spring 2021}

\begin{document}

\maketitle

\begin{frame}{Introduction}

\begin{itemize}
	\item \textbf{Objective:} Aim a trebuchet simulator to hit a specified distance.
	\item The trebuchet is aimed by adjusting three parameters:
		\begin{enumerate}
			\item Fulcrum height
			\item Counterweight mass
			\item Sling length
		\end{enumerate}
	\item Each team will receive 12 training runs to build a model that predicts the optimal trebuchet settings.
	\item The average error for 8 testing runs will determine your group's final score.
	\item The simulator is stochastic --- even if you find the optimal parameter settings, you will not hit the target exactly.
		\begin{itemize}
			\item The stochasticity is controlled, i.e.\ each team will receive the same sequence of random deviates.
		\end{itemize}
	\item You will have access to a demo simulator to refine your methods. The final simulator will use similar parameter ranges but will not behave identically.
	\item On \textbf{Monday, April~5}, the final simulator will open for 48~hours. Slides describing your strategy and results are due \textbf{Friday, April~9}.
\end{itemize}

\end{frame}

\begin{frame}{Rules}
\begin{enumerate}
	\item Teams of 2--3 are allowed, but optional.
	\item No sharing data across teams.
	\item Your parameter settings must be derived from your model \textbf{for all testing runs}. No tweaking without guidance from the model!
	\item You can use the demo simulator as many times as you want. Your group only has one chance with the final simulator.
	\item Grading is based on your process, with bonus points for results.
	\item Members of the winning team are allowed to claim ``BIOE Trebuchet Champions 2021'' on their CV/resume.
\end{enumerate}	
\end{frame}


\end{document}
