\documentclass[pal,wide]{pajarticle}

\title{Exam 1 Topics}
\author{BIOE 498/598 PJ}
\date{Spring 2021}

\begin{document}

\maketitle

\newcommand\Vocab{\textbf{Vocab:}}

\begin{enumerate}
	\item Use the bootstrap to build a null distribution and calculate a $p$-value.
	\item Use and interpret the results of a $t$-test.
	\item \Vocab\ response, predictor, factor, intercept, coefficient, effect size, parameter, residual.
	\item Use effect sizes to relate changes in factor levels to changes in the response.
	\item Use linear models for hypothesis testing.
	\item Explain the meaning of interactions.
	\item Calculate the number of interactions in a model with $n$ factors.
	\item Explain how transformations affect the relationship between factors and response.
	\item Transformations: mean centering, $z$-scoring, rescaling to compare binary and continuous factors.
	\item Apply and interpret the results of a Box-Cox analysis.
	\item \Vocab\ run, experiment, experimental unit, replicate, duplicate, background variable, effect, experimental design, confounded factors, biased factors, bias error, random error.
	\item Explain the differences between continuous, ordinal, and nominal factors.
	\item Apply one-hot encoding to nominal factors.
	\item Explain why degeneracy arises in models with an intercept and multilevel factors.
	\item Define and interpret contrasts.
	\item Determine if a contrast is estimable.
	\item Understand and apply blocking factors.
	\item Explain the advantages and disadvantages of factorial designs.
	\item Calculate the number of runs for a factorial design.
	\item Find the degrees of freedom in a model.
	\item Explain and interpret half-normal plots.
	\item \Vocab\ effect sparsity principle, hierarchical ordering principle, heredity principle.
	\item \Vocab\ practical and statistical significance.
	\item Fractional Factorial Designs
	\begin{itemize}
		\item Use generators to derive the defining relation.
		\item Use the defining relation to compute confounding structure.
		\item Compute and interpret the resolution, aberration, and clarity of a design.
		\item Use foldover and mirror image designs to clear confounded factors.
	\end{itemize}
	\item Plackett-Burman Designs
	\begin{itemize}
		\item Construct PB designs for a set number of factors.
		\item \Vocab\ complex aliasing, hidden projection property.
		\item Explain how to fit a PB design with a linear model.
		\item Interpret the results of subset selection.
	\end{itemize}
	\item \Vocab\ mixed-level factorial designs and Orthogonal Arrays.
	\item Interpret the 95\% CI for effects in a model.
	\item Perform power analysis (standard normal and $t$-test) on model coefficients.
	\item Explain the limitations of power analysis.
	\item ANOVA
	\begin{itemize}
		\item Explain the decomposition of the sum of squares for a model.
		\item Compute $SS_\text{total}$, $SS_\text{explained}$, $SS_\text{residual}$, and the degrees of freedom for each.
		\item Compute the $F$ statistic for an entire model and an individual factor.
	\end{itemize}
\end{enumerate}

\end{document}